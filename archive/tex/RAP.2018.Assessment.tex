

%%---------
% Use master copy at bio.snopwcrab/inst/tex !!!!
%%----------------- 

%----------------------------------------------------------------------------------------
%	Presentation for snow crab RAP using beamer
% last edit: Feb 2018
%
% Note: to reduce the size of the PDF:
% use ghostscript:
% gs -q -dSAFER -dNOPAUSE -sDEVICE=pdfwrite -sPDFSETTINGS=printer -sOutputFile="resdoc2006sc.pdf" resdoc2006sc.pdf 
% the key being sPDFSETTINGS: with options: default, screen, ebook, printer, preprint
% try also:  -sCompressPages=true
% -sDownsampleColorImages=true
% -sColorImageResolution=300
% -sGrayImageResolution=300
% -sMonoImageResolution=300
% final choice to get it under 8MB :
% gs -q -dSAFER -dNOPAUSE -sDEVICE=pdfwrite -dPDFSETTINGS=/printer -dColorImageResolution=150 -dMonoImageResolution=150 -dGrayImageResolution=150  -dCompatibilityLevel=1.4 -sOutputFile="resdoc2006sc.printer.pdf" resdoc2006sc.pdf  < /dev/null
%----------------------------------------------------------------------------------------


\documentclass{beamer}

\mode<presentation> {
  %\usetheme{Hannover}
  %\usetheme{AnnArbor}
  \usetheme{Boadilla}
  %\usecolortheme{dolphin}
  \usecolortheme{seagull}
  \usefonttheme{structuresmallcapsserif}
  \beamertemplatenavigationsymbolsempty % turn off navigation
  \hypersetup{pdfstartview={Fit}} % fits the presentation to the window when first displayed
}

%\usepackage{default}
%\usepackage{fourier}
%\usepackage[english]{babel}															% English language/hyphenation
%\usepackage[protrusion=true,expansion=true]{microtype}				% Better typography
%\usepackage[toc,page]{appendix}
%\usepackage[utf8]{inputenc}
%\usepackage{csquotes}
%\usepackage{hyperref}
\usepackage{graphicx} % Allows including images
%\usepackage{graphics}
%\usepackage{booktabs} % Allows the use of \toprule, \midrule and \bottomrule in tables
%\usepackage{amsmath,amsfonts}				% Math packages
\numberwithin{equation}{section}		% Equationnumbering: section.eq#
\numberwithin{figure}{section}	   	% Figurenumbering: section.fig#
\numberwithin{table}{section}				% Tablenumbering: section.tab#



%----------------------------------------------------------------------------------------
% these need to be incremented each year .. <<<<<<<<<<<<<<<<<<<<<
\newcommand{\yr}{2017}
\newcommand{\yrminusone}{2016}
\newcommand{\yrminustwo}{2015}
\newcommand{\yrminusthree}{2014}
\newcommand{\yrminusfour}{2013}
\newcommand{\yrminusfive}{2012}
% these need to be incremented each year .. <<<<<<<<<<<<<<<<<<<<<


% ----------------------------------
\newcommand{\D}{.}  % to replace dots .. 
\newcommand{\h}{C:/} %home directory
\newcommand{\e}{bio.data/}
\newcommand{\es}{bio.data/bio.snowcrab/} %snow crab assessment base directory
\newcommand{\eb}{bathymetry/}
\newcommand{\ea}{bio.data/aegis/}
\newcommand{\Ay}{assessments/2017/}
\newcommand{\A}{assessments/}
\newcommand{\m}{output/maps/} %base folder for maps
\newcommand{\mb}{output/maps/survey/snowcrab/annual/bycatch/} %survey bycatch folder


% ----------------------------------
\title[Snow Crab Assessment \yr]{Snow Crab Assessment\\  Maritimes Region\\ \yr } 
\author[Snow Crab Unit]{Ben Zisserson, Brent Cameron, Amy Glass, Jae Choi} 
\institute[DFO Science]{
  Canadian Department of Fisheries and Oceans \\ % Your institution for the title page
  Science Branch \\
  Population Ecology Division
  \medskip
  \textit{} % Your email address
}
 

% ----------------------------------
\begin{document}


% ----------------------------------
  \begin{frame}
    \titlepage % Print the title page as the first slide
  \end{frame}
  

% ----------------------------------
  \begin{frame}
    \frametitle{Overview} % Table of contents slide
    \tableofcontents % \section{} and \subsection{} 
  \end{frame}
  
\section{Survey}
%\subsection{Sampling}  
% ----------------------------------
  \begin{frame}
    \frametitle{ENS Snow Crab}
    \begin{columns}
      \begin{column}{0.4\textwidth}
        \begin{itemize}
        
        	\item Southern limit of Atlantic Snow Crab distribution 
        	\item Three assessment areas
        	\item Four fishery management areas
        \end{itemize}
      \end{column}
      
      \begin{column}{0.6\textwidth}
      	\begin{centering}
          \begin{figure}
            \includegraphics[width=\textwidth]{\h \es \A /common/Basemap.png}
          \end{figure}
      	\end{centering}
      \end{column}
    \end{columns}
  \end{frame}
  

% ----------------------------------
%
%\iffalse % comments out following code until reaching \fi

%\subsection{Maps}
%


\begin{frame}
\frametitle{Survey}
 
  	\begin{columns}
  		\begin{column}{0.4\textwidth}
  		 \begin{itemize}
  		 	\item Started in late 1990's by GFC
  		 	\item Standardized sampling since 2004
  		 	\item Specialized trawl designed to dig into bottom sediments
  		 	\item Habitat, community and biological data collected
  		 	\item 32 stations not sampled in 2017 due to weather and time constraints %survey year specific comment(s)
  		 \end{itemize}
  		\end{column}
  		
  		\begin{column}{0.6\textwidth}
  			\begin{centering}
  					\begin{figure}
  						\includegraphics[width=0.7\textwidth]{\h \es \m survey\D locations/survey\D locations\D \yrminusone.png}\\   
  						\includegraphics[width=0.7\textwidth]{\h \es \m survey\D locations/survey\D locations\D \yr.png}\\   
  					\end{figure}
  		
  			\end{centering}
  		\end{column}
  	\end{columns}
  		
  \end{frame}
%\fi

\begin{frame}
\frametitle{Survey Timing}
 \vspace*{0.4cm}
Day of year from August 15
 \vspace*{-0.65cm}
\begin{columns}[T]
	\begin{column}{0.5\textwidth}
		\begin{figure}
			\includegraphics[width=0.75\textwidth]{\h \es \m survey/snowcrab/annual/julian\D compressed/julian\D compressed\D \yrminusfive.png}\\   
			\includegraphics[width=0.75\textwidth]{\h \es \m survey/snowcrab/annual/julian\D compressed/julian\D compressed\D \yrminusone.png}\\   
		\end{figure}
	\end{column}
	
	\begin{column}{0.5\textwidth}
		\begin{centering}
			\begin{figure}
				\includegraphics[width=0.75\textwidth]{\h \es \m survey/snowcrab/annual/julian\D compressed/julian\D compressed\D \yrminusthree.png}\\   
				\includegraphics[width=0.75\textwidth]{\h \es \m survey/snowcrab/annual/julian\D compressed/julian\D compressed\D \yr.png}\\   
			\end{figure}
		\end{centering}
	\end{column}
\end{columns}

\end{frame}




% ----------------------------------
\section{Ecosystem Interactions}
\begin{frame}
\frametitle{Community Interactions}

\begin{itemize}
	\setlength\itemsep{2em}
	\item Competitors: Some benthic fish and crabs, but few strong competitors
	\item Prey: Echinoderms, shrimp, crabs, worms, bivalves, sea stars
	\item Predators: Halibut, Wolfish, Skates, Longhorn Sculpin, Atlantic Cod, White Hake, Plaice, Haddock, other Snow Crab
\end{itemize}
\end{frame}

%------------------------------------------------------------------------------
%\subsection{Competition/ Prey}
\begin{frame}
\frametitle{Competition/Prey}
Competitors/prey (kg\textbackslash$km^2$) of Snow Crab in the Snow Crab Survey
	\begin{columns}
	\begin{column}{0.2\textwidth}
	\begin{itemize}
	  \setlength\itemsep{2em}
		\item[] Jonah Crab 
		%\item[]
		\item[] Lesser Toad Crab
		%\item[]
		\item[] Northern Shrimp
	\end{itemize}
	\end{column}

	\begin{column}{0.35\textwidth}
 	\begin{figure}
    \includegraphics[width=0.75\textwidth,trim={0 1cm 0 1cm}]{\h \es \mb ms\D mass\D 2511/ms\D mass\D 2511\D \yrminusone.png}\\   
    \includegraphics[width=0.75\textwidth,trim={0 1cm 0 1cm}]{\h \es \mb ms\D mass\D 2521/ms\D mass\D 2521\D \yrminusone.png}\\   
    \includegraphics[width=0.75\textwidth,trim={0 1cm 0 1cm}]{\h \es \mb ms\D mass\D 2211/ms\D mass\D 2211\D \yrminusone.png}  
  	\end{figure}
  	\end{column}

  	\begin{column}{0.35\textwidth}
 		\begin{figure}
			\includegraphics[width=0.75\textwidth,trim={0 1cm 0 1cm}]{\h \es \mb ms\D mass\D 2511/ms\D mass\D 2511\D \yr.png}\\   
			\includegraphics[width=0.75\textwidth,trim={0 1cm 0 1cm}]{\h \es \mb ms\D mass\D 2521/ms\D mass\D 2521\D \yr.png}\\   
			\includegraphics[width=0.75\textwidth,trim={0 1cm 0 1cm}]{\h \es \mb ms\D mass\D 2211/ms\D mass\D 2211\D \yr.png}  
\end{figure} 
  	\end{column}

  	\end{columns}
\end{frame}


% ----------------------------------
%\subsection{Predation}
\begin{frame}
\frametitle{Potential Predation}
Potential predators (kg\textbackslash$km^2$) of snow crab in the Snow Crab Survey: \textbf{Atlantic Cod}
	\begin{columns}

	\begin{column}{0.4\textwidth}
 	\begin{figure}
    \includegraphics[width=0.9\textwidth]{\h \es \mb ms\D mass\D 10/ms\D mass\D 10\D \yrminusone.png}\\   
    \includegraphics[width=0.9\textwidth]{\h \es \mb ms\D mass\D 10/ms\D mass\D 10\D \yr.png}\\   
    \end{figure}
  	\end{column}

	\begin{column}{0.6\textwidth}
 	\begin{figure}
  	\includegraphics[width=0.8\textwidth]{\h \es \Ay timeseries/survey/ms\D mass\D 10.pdf}\\
   	\end{figure}
  	\end{column}

  	\end{columns}
\end{frame}


% ----------------------------------
\begin{frame}
\frametitle{Potential Predation}
Potential predators (kg\textbackslash$km^2$) of snow crab in the Snow crab survey: \textbf{Thorny Skate}
\begin{columns}
	
	\begin{column}{0.4\textwidth}
		\begin{figure}
			\includegraphics[width=0.9\textwidth]{\h \es \mb ms\D mass\D 201/ms\D mass\D 201\D \yrminusone.png}\\   
			\includegraphics[width=0.9\textwidth]{\h \es \mb ms\D mass\D 201/ms\D mass\D 201\D \yr.png}\\   
		\end{figure}
	\end{column}
	
	\begin{column}{0.6\textwidth}
		\begin{figure}
			\includegraphics[width=0.8\textwidth]{\h \es \Ay timeseries/survey/ms\D mass\D 201.pdf}\\
		\end{figure}
	\end{column}
	
\end{columns}
\end{frame}


%--------------------------------------------------------------
\section{Resource Status}
%\subsection{Reproductive Potential}


% ----------------------------------
\begin{frame}
\frametitle{Female Size-Frequency}
\begin{columns}
	\begin{column}{0.5\textwidth}
		\begin{itemize}
		\item Size-frequency histograms of carapace width of female Snow Crab
		\item Majority of females now mature
		\item Expect increasing egg production for next 1-3 years
		\end{itemize}
	\end{column}

	\begin{column}{0.5\textwidth}
	\begin{figure}
		\includegraphics[width=\textwidth]{\h \es \Ay figures/size\D freq/survey/female.pdf}
	\end{figure}
	\end{column}
	\end{columns}
\end{frame}



% ----------------------------------
\begin{frame}
\frametitle{Primiparous/Multiparous}
\begin{columns}
	\begin{column}{0.5\textwidth}
	\begin{center}
	Primiparous 
	\vspace*{-0.65cm}
	\end{center}
		\begin{figure}
		\includegraphics[width=\textwidth]{\h \es \m survey/snowcrab/annual/totno\D female\D primiparous/totno\D female\D primiparous\D \yr.png}
	\end{figure}
	\end{column}

	\begin{column}{0.5\textwidth}
	\begin{center}
	Multiparous 
	\vspace*{-0.65cm}
	\end{center}
	\begin{figure}
		\includegraphics[width=\textwidth]{\h \es \m survey/snowcrab/annual/totno\D female\D multiparous/totno\D female\D multiparous\D \yr.png}	
	\end{figure}
	\end{column}
\end{columns}
\end{frame}



% ----------------------------------
\begin{frame}
\frametitle{Mature Female Abundance}

	\begin{columns}
    \begin{column}{0.5\textwidth}
      \begin{figure}
        \includegraphics[width=\textwidth]{\h \es \m survey/snowcrab/annual/totno\D female\D mat/totno\D female\D mat\D \yr.png}	
      \end{figure}
    \end{column}

  	\begin{column}{0.5\textwidth}
      \begin{figure}
        \includegraphics[width=0.9\textwidth]{\h \es \Ay timeseries/survey/totno\D female\D mat.pdf}
      \end{figure}
    \end{column}
  \end{columns}
Reproductive potential peaked in 2007/8 (2012 in 4X)
\end{frame}

\begin{frame}
\frametitle{Mature Sex Ratios}

\begin{columns}
	\begin{column}{0.3\textwidth}
		\begin{itemize}
			\begin{footnotesize}
				\item  SSE: Traditionally male dominated 
				\item S-ENS: Stable
				\item N-ENS: Approaching 50\%. Recovering from near zero
				\item 4X: Continued high proportion of males. Overall numbers very low  
			\end{footnotesize}
		\end{itemize}
	\end{column}
	
	\begin{column}{0.7\textwidth}
		\begin{figure}
			%	\includegraphics[width=	0.6\textwidth]{\h \es \m survey/snowcrab/annual/sexratio\D mat/sexratio\D mat\D \yr.png}\\
			\includegraphics[width=0.8\textwidth]{\h \es \Ay timeseries/survey/sexratio\D mat.pdf}
		\end{figure}
	\end{column}
	
\end{columns}
\end{frame}


% ----------------------------------
%\subsection{Recruitment}
\begin{frame}
\frametitle{Recruitment}
\begin{columns}
	\begin{column}{0.5\textwidth}
		Size-frequency histograms of carapace width of male Snow Crab. The vertical line represents the legal size (95 mm)
		\begin{itemize}
			\item S-ENS: Moderate internal recruitment
			\item N-ENS: Leading edge of recruitment pulse now entering fishery. Protecting soft crab essential.
			\item 4x: Minimal internal recruitment for the foreseeable future  
		\end{itemize}
	\end{column}
	
	\begin{column}{0.5\textwidth}
		\begin{figure}
			\includegraphics[width=\textwidth]{\h \es \Ay figures/size\D freq/survey/male.pdf}
		\end{figure}
	\end{column}
\end{columns}
\end{frame}




% ----------------------------------
\section{Space Time Modelling}
\begin{frame}
\frametitle{Space Time Modelling --  lbm \& stmv}

\begin{itemize}
	\item Hierarchal process
	\item Global model of environmental factors upon snow crab
	\item Location specific timeseries analysis with seasonal and annual harmonic components
	\item Local spatial interpolation of variability at each location and time
	\item Results combine to form prediction surfaces for snowcrab habitat and abundance
\end{itemize}
\end{frame}

% ----------------------------------

\begin{frame}
\frametitle{Space Time Modelling -- lbm vs stmv }


		\vspace*{-.5cm}
		\begin{itemize}
		\item stmv is refinement of 2016 lbm approach
		\item stmv removed some model inputs (metabolic rate, condition index, etc)
		\item Added localized temporal smoothing
		\item More stable inter-annual estimates
		\item More stable results allow biomass dynamics model to be more informative
		\item New approach requires recalculation of past results
		\end{itemize}


\end{frame}


%------------------------------------------------------------------------------
%\subsection{Ecosystem Inputs -- Environmental}
\begin{frame}
\frametitle{Ecosystem Inputs -- Environmental}
	\begin{columns}
	\begin{column}{0.25\textwidth}
	\begin{itemize}
	  \setlength\itemsep{2em}
		\item[] Bottom Temperature 
		\item[] Bottom Temperature SD
		\item[] Substrate ln(grain size)
	\end{itemize}
	\end{column}

	\begin{column}{0.25\textwidth}
 	\begin{figure}
    \includegraphics[width=\textwidth]{\h \ea temperature/maps/SSE/bottom\D predictions/climatology/temperatures\D bottom.png}\\   
    \includegraphics[width=\textwidth]{\h \ea temperature/maps/SSE/bottom\D predictions/climatology/temperatures\D bottom\D sd.png}\\   
    \includegraphics[width=\textwidth]{\h \ea substrate/R/substrate\D grainsize\D edit.png}  
  	\end{figure}
  	\end{column}

  	\begin{column}{0.25\textwidth}
 	\begin{figure}
    \includegraphics[width=\textwidth]{\h \ea \eb maps/SSE/depth.png}\\   
    \includegraphics[width=\textwidth]{\h \ea \eb maps/SSE/slope.png}\\   
    \includegraphics[width=\textwidth]{\h \ea \eb maps/SSE/curvature.png}  
  	\end{figure}
  	\end{column}

	\begin{column}{0.25\textwidth}
	\begin{itemize}
	  \setlength\itemsep{2em}
		\item[] Depth
		\item[] 
		\item[] Slope
		\item[] 
		\item[] Curvature
	\end{itemize}
  	\end{column}

  	\end{columns}
\end{frame}


\begin{frame}
\frametitle{Ecosystem Inputs -- Temperature}
\vspace*{-0.4cm}
	\begin{columns}

	\begin{column}{0.5\textwidth}
 	\begin{figure}
    \includegraphics[width=0.55\textwidth]{\h \ea temperature/maps/SSE/bottom\D predictions/annual/temperatures\D bottom\D  1960.png}\\   
    \includegraphics[width=0.55\textwidth]{\h \ea temperature/maps/SSE/bottom\D predictions/annual/temperatures\D bottom\D  2012.png}\\   
    \includegraphics[width=0.55\textwidth]{\h \ea temperature/maps/SSE/bottom\D predictions/annual/temperatures\D bottom\D  \yrminusone.png}\\   
  	\end{figure}
  	\end{column}

  	\begin{column}{0.5\textwidth}
 	\begin{figure}
    \includegraphics[width=0.55\textwidth]{\h \ea temperature/maps/SSE/bottom\D predictions/annual/temperatures\D bottom\D  2000.png}\\   
    \includegraphics[width=0.55\textwidth]{\h \ea temperature/maps/SSE/bottom\D predictions/annual/temperatures\D bottom\D  \yrminusthree.png}\\   
    \includegraphics[width=0.55\textwidth]{\h \ea temperature/maps/SSE/bottom\D predictions/annual/temperatures\D bottom\D  \yr.png}\\   
  	\end{figure}
  	\end{column}

 	\end{columns}
\end{frame}

%\subsection{Ecosystem Inputs -- Community}
\begin{frame}
\frametitle{Ecosystem Inputs -- Community Structure}

	\begin{columns}
	\begin{column}{0.2\textwidth}
	\begin{center}
	\begin{itemize}
	  \setlength\itemsep{2em}
		\item[] PCA 1st Axis 
		\item[]
		\item[] PCA 2nd Axis
	\end{itemize}
	\end{center}
	\end{column}

	\begin{column}{0.7\textwidth}
 	\begin{figure}
   		\includegraphics[width=0.55\textwidth,trim={0 1cm 0 1cm}]{\h \ea speciescomposition/maps/pca1/snowcrab/annual/pca1\D mean\D 2016.png}\\
  		\includegraphics[width=0.55\textwidth,trim={0 1cm 0 1cm}]{\h \ea speciescomposition/maps/pca2/snowcrab/annual/pca2\D mean\D 2016.png}\
  	\end{figure}
  	\end{column}

  	\end{columns}
\end{frame}



%\subsection{Fishable Biomass STMV}
\begin{frame}
\frametitle{Fishable Biomass Index -- STMV}

\begin{columns}

\begin{column}{0.5\textwidth}
  \begin{center}
    Habitat Space
  \end{center}
  \vspace*{-0.6cm}
  	\begin{figure}
    \includegraphics[width=0.7\textwidth,trim={0 1cm 0 1cm}]{\h \es maps/snowcrab\D large\D males_presence_absence/snowcrab/annual/snowcrab\D large\D males_presence_absence\D mean\D \yr.png}
  	\end{figure}
\end{column}

\begin{column}{0.5\textwidth}
  \begin{center}
    Survey Abundance Index 
  \end{center}
\vspace*{-0.6cm}
  \begin{figure}
    \includegraphics[width=0.7\textwidth,trim={0 1cm 0 1cm}]{\h \es maps/snowcrab\D large\D males_abundance/snowcrab/annual/snowcrab\D large\D males_abundance\D mean\D \yr.png}
  \end{figure}
\end{column}

\end{columns}

\vspace*{0.2cm}
\begin{center}
Fishable Biomass Index
\end{center}
\vspace*{-0.2cm}
\begin{figure}
	\includegraphics[width=0.45\textwidth,trim={0 1cm 0 1cm}]{\h \es maps/fishable.biomass/snowcrab/prediction\D abundance\D mean\D \yr.png}
\end{figure}
	
\end{frame}


%------------------------------------------------
%\begin{frame}
 % \frametitle{Fishable Biomass Index -- STMV}
 
    %  \begin{figure}
    %    \includegraphics[width=0.8\textwidth]{\h \es maps/fishable.biomass/snowcrab/prediction\D abundance\D mean\D \yr.png}
    %  \end{figure}
  

%\end{frame}

% ----------------------------------
\begin{frame}
\frametitle{Precautionary Approach}

		\begin{itemize}
			\item Guiding principle for Fisheries Management
			\item Attempts to make careful decisions accounting for uncertainties in natural systems
			\item Canada has a legally binding international obligation to manage NR using a "PA"
			\item Establishes biological reference points and defines management actions around these reference points
			\item Assumes management actions are a dominant force in stock dynamics
			\item SSE snow crab fishery is inherently precautious
				\begin{itemize}
				\item no removal of females
				\item only mature males removed
				\item relatively low exploitation
				\item catches monitored (at-sea and dockside)
				\end{itemize}
		\end{itemize}

\end{frame}
%------------------------------------------------

\begin{frame}
\frametitle{Precautionary Approach}
\begin{figure}[ht]
	\centering
	\includegraphics[width=0.8\textwidth]{\h \es \A common/hcr_ideal.pdf}
\end{figure}
Suite of biological indicators inform decisions within "Target Harvest Area"
\end{frame}


%------------------------------------------------
\section{Fisheries Model}
\begin{frame}
\frametitle{Fishable Biomass -- Biomass dynamics}
	\begin{itemize}
		\item Use fisheries model to determine relevant biological reference points
		\item Unable to use classic fishery models
			\begin{itemize}
				\item aging not possible
				\item complex life cycles
				\item spatially and temporally variable size  and sex structure
			\end{itemize}
		\item Use a biomass dynamics ("Surplus Production") model
		\item Provides estimates of posterior distributions of parameters (K, r, q, F, Fmsy) 
	\end{itemize}
\end{frame}


% ----------------------------------
%\subsection{Posteriors}
\begin{frame}
\frametitle{Posteriors -- Carrying Capacity}
\begin{figure}
	\centering
	\includegraphics[width=0.5\textwidth]{\h \es \Ay K\D density.png} 
\end{figure}
Attempts to define the maximum population size an environment can sustain indefinitely
\end{frame}


% ----------------------------------
\begin{frame}
\frametitle{Posteriors -- Population Growth Rate}
\begin{figure}
\centering
\includegraphics[width=0.5\textwidth]{\h \es \Ay r\D density.png} 
\end{figure}
Inherently includes natural mortality, immigration, growth, incidental fishery mortality 
\end{frame}


% ----------------------------------
\begin{frame}
\frametitle{Posteriors -- Catchability}
\begin{figure}
\centering
\includegraphics[width=0.5\textwidth]{\h \es \Ay q\D density.png} 
\end{figure}
Simplistically quantifies the influence of differing biases such as survey gear and sampling,  areal expansion,  statistical modeling, etc.
\end{frame}


% ----------------------------------
%\subsection{Fishable Biomass -- Biomass dynamics}
\begin{frame}
\frametitle{Fishable Biomass -- Biomass dynamics}
\begin{columns} 
\begin{column}{0.4\textwidth}
 \begin{itemize}
  \begin{tiny}
     \item Red: The stmv fishable biomass index, Green : q-corrected
     \item Blue: The posterior mean fishable biomass estimated from the logistic model
     \item Grey: The density distribution of posterior fishable biomass with estimates darkest area being medians 
     \end{tiny}
    \end{itemize}
\end{column}
\begin{column}{0.6\textwidth}
  \begin{figure}[ht]
    \centering
    \includegraphics[width=0.9\textwidth]{\h \es \Ay biomass\D timeseries.png}
  \end{figure}
\end{column}
\end{columns} 

\end{frame}


%------------------------------------------------
%\subsection{Fishing Mortality}
\begin{frame}
\frametitle{Fishing Mortality}
\begin{columns}
\begin{column}{0.4\textwidth}

\begin{itemize}
\begin{footnotesize}
	\item N-ENS: Less stable but similar mean to S-ENS
	\item S-ENS: Relatively stable over past 15 years
	\item 4x: High since 2011
\end{footnotesize}
\end{itemize}
\end{column}

\begin{column}{0.6\textwidth}


\begin{figure}
    \includegraphics[width=0.85\textwidth]{\h \es \Ay fishingmortality\D timeseries.png}
\end{figure}
\begin{itemize}
	\begin{tiny}
		\item[] Grey Area: Posterior density distributions, with median  
		\item[] Dark-dashed line: is the 20\% harvest rate
		\item[] Red line: estimated FMSY 	
	\end{tiny}
\end{itemize}

\end{column}

\end{columns}
\end{frame}


% ----------------------------------
%\begin{frame}
%  \frametitle{Posteriors -- Process Errors (sd)}
%  \begin{figure}
%    \centering
%    \includegraphics[width=0.6\textwidth]{\h \es \Ay bp\D sd\D density.png} 
%  \end{figure}
%\end{frame}


% ----------------------------------
%\begin{frame}
%  \frametitle{Posteriors -- Observation Errors (sd)}
%  \begin{figure}
%    \centering
%    \includegraphics[width=0.6\textwidth]{\h \es \Ay bo\D sd\D density.png} 
%  \end{figure}
%\end{frame}

%\subsection{Harvest Advice}

% ----------------------------------


% ----------------------------------
\begin{frame}
  \frametitle{Precautionary approach}
\begin{figure}
  \centering
  \includegraphics[width=0.6\textwidth]{\h \es \Ay hcr\D default.png}\\ 
\end{figure}
\end{frame}

\begin{frame}
\frametitle{Harvest Advice}

\begin{itemize}
	\item N-ENS
	\begin{itemize}
		\item Refined assessment methodology stabilized inter-annual variabity in biomass estimates
		\item Stock is in the "heathly zone" though below long-term mean
		\item A moderate decrease in TAC is recommended
	\end{itemize}
	\item S-ENS
	\begin{itemize}
		\item Refined assessment methodology shows more stable trends of Fishing Mortality and Biomass
		\item Stock is in the "heathly zone" though below long-term mean
		\item Biomass index continued to decline in spite of TAC reductions
		\item A moderate decrease in TAC is recommended
	\end{itemize}
	\item 4X
	\begin{itemize}
		\item Low recruitment and poor environmental conditions create uncertainties about this population
		\item Stock is in the "critical zone"
		
	\end{itemize}
\end{itemize}
\end{frame}

% ----------------------------------
% ----------------------------------
% ----------------------------------

\end{document}

